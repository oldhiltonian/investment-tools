\documentclass{article}
\usepackage{amsmath} % for equations
\usepackage{xcolor}
\usepackage[top=0.5in, bottom=0.5in, left=0.5in, right=0.5in]{geometry}

\title{Introduction to Fundamental Analysis}
\author{John Forrester Davidson}
\date{} % leave date blank

\begin{document}
\maketitle
\tableofcontents

\section{Introduction}

These notes aim to cover the fundamental aspects of financial analysis. The focus will be on the balance sheet, income statement, and cash flow statements, which are the main financial statements used to evaluate a company's financial performance and position. Additionally, there will be sections on stocks and bonds, which are the main types of securities that companies issue to raise capital. These sections will cover what stocks and bonds represent, and how to value these securities. The objective of these notes is to provide a comprehensive understanding of the key concepts and tools used in financial analysis, which will enable the reader to effectively analyze a company's financial statements and make informed investment decisions.

\section{Balance Sheets}

We begin by looking at the balance sheet, which is a snapshot of a company's financial position at a single point in time. It reports the company's assets, liabilities, and equity as of a specific date. It is one of the most important financial statements for understanding a company's overall financial health. It helps to understand the company's liquidity, its ability to pay off its debt, and the amount of assets it owns compared to how much it owes. The balance sheet provides a detailed picture of what a company owns (assets) and what it owes (liabilities) at a specific point in time. By comparing the balance sheet of a company from one period to another, you can understand how the company's financial position has changed over time. The basic accounting equation to consider is the following: 

\begin{equation}
    Assets = Liabilities + Owners Equity
\end{equation}

The intuition behind this equation is as follows. A company's assets are financed either by borrowing money (liabilities) or by using the funds of its owners (equity). In other words, the assets of a company are the result of the funding provided by its creditors and its owners. Assets represent the resources the business has available to generate income and grow the business. Liabilities, on the other hand, represent the obligations or debts that the company has to its creditors. Owners' equity represents the residual interest in the assets of the company after liabilities are deducted, it represents the portion of the assets that the company's owners can claim as their own. It helps to understand that the assets of the company are the means to generate revenue, the liabilities are the obligations to pay back to others and the owners equity are the funds that the owners have invested in the company and what they are left with.

\subsection{Assets}

Long-term assets and current assets are two types of assets that are typically found on a company's balance sheet.\\

Current assets are assets that are expected to be converted into cash or used up within one year of the balance sheet date. Examples of current assets include cash, accounts receivable, and inventory. These assets are considered to be liquid, as they can be easily converted into cash to pay off short-term debts or meet other short-term obligations.\\

Long-term assets, on the other hand, are assets that are not expected to be converted into cash or used up within one year of the balance sheet date. Examples of long-term assets include property, plant, and equipment (PPE), investments in other companies, and intangible assets such as patents and trademarks. These assets are considered to be less liquid than current assets, as they are not as easily converted into cash. Note that inventory may under some circumstances take a while to be sold, and if liquidation occurs then often inventory is sold as a steep discount. \\

Property, Plant and Equipment (PPE) is a type of long-term asset that includes tangible assets used in the operations of a business, such as land, buildings, machinery, vehicles and office equipment. This type of assets are used in the production of goods or provision of services and are not intended for resale. PPE typically has a useful life of more than one year, and is recorded at cost and depreciated over time.

\subsection{Liabilities}

The liabilities section of a balance sheet lists a company's outstanding debts and financial obligations as of a specific date. It represents the money that the company owes to its creditors and other parties. The liabilities section is typically divided into two categories: current liabilities and long-term liabilities.\\

Current liabilities are debts and obligations that are due within one year of the balance sheet date. Examples of current liabilities include accounts payable, short-term loans, and taxes owed. These are debts that the company expects to pay off within the next 12 months using current assets such as cash, accounts receivable, and inventory.\\

Long-term liabilities are debts and obligations that are due beyond one year of the balance sheet date. Examples of long-term liabilities include long-term loans, bonds, and leases. These are debts that the company expects to pay off over a period of time greater than one year, typically with the help of long-term assets such as PPE and investments.\\

It's important to note that the liabilities section of the balance sheet not only includes debt but also any other financial obligations. For example, it may include things like rent or salaries that are not yet paid, or taxes that are owed.\\

The liabilities section of the balance sheet is important for understanding a company's financial position and its ability to pay off its debts. A company with a lot of liabilities may have a harder time paying off its debts than a company with fewer liabilities. Additionally, the liabilities section can provide insight into the company's financial strategy and risk management policies. A company with a lot of short-term liabilities may be taking on more risk than a company with mostly long-term liabilities.

\subsection{Owners Equity}
The owners' equity portion of a balance sheet represents the residual interest in the assets of the company after deducting liabilities. It represents the value of the assets that the company's shareholders would be entitled to if the company were to be liquidated and all of its assets sold and all of its liabilities paid off. It is also known as stockholders' equity, it is the value of the shareholders' interest in the company.

The owners' equity section of the balance sheet typically includes the following accounts:

\begin{itemize}
    \item Common Stock: This represents the number of shares of stock a company has issued to its shareholders and the par value of those shares.
    \item Paid-in Capital: This represents the amount of money shareholders have invested in the company in exchange for stock, in excess of the par value.
    \item Retained Earnings: This represents the portion of net income that the company has chosen to retain and not distribute as dividends.
    \item Treasury Stock: This represents the shares of stock that a company has bought back from its shareholders.
    \item Accumulated Other Comprehensive Income: This represents other income and expenses that are not included in net income, such as foreign currency translation adjustments or unrealized gains or losses on securities.
\end{itemize}

The owners' equity section of the balance sheet is important for understanding a company's financial position and its ability to raise capital. It can provide insight into how much money shareholders have invested in the company and how much the company has earned over time. Additionally, it can provide insight into a company's dividend policy and its approach to stock buybacks.

Note that owners equity is not a liability, as it is not owed to anybody except in the sense that the company is "owed to" or belongs to the shareholders. 

\subsubsection{A Note on Book Value}
Book value is a financial metric that is calculated by taking the total assets of a company and subtracting its total liabilities. It represents the value of a company's assets if they were to be liquidated at their current market value.

In some cases, the book value of a company can be equal to its owners' equity. This typically happens when a company has few liabilities and its assets are valued at their historical cost rather than their market value. However, this is not always the case. The book value can be different from the owners' equity, especially when the company's assets are valued at market value rather than historical cost, this will cause the book value to be different from the owners' equity.

It's important to note that Book value and owners' equity are not the same thing, but they are related. The book value is a measure of the company's net asset value, whereas the owner's equity is a measure of the shareholder's stake in the company.



\section{Income Statements}

An income statement, also known as a profit \& loss statement (P\&L), is a financial statement that shows a company's revenues and expenses over a specific period of time, typically a fiscal quarter or year. The income statement is used to calculate a company's net income, which is the difference between its total revenue and total expenses. It helps to understand how much a company is making or losing over a period of time. An income statement typically includes the following sections:

\begin{itemize}
    \item Revenues: This section includes all the money a company earns from selling goods or services. It includes both cash and credit sales.

    \item Cost of goods sold (COGS): This section includes all the costs directly associated with producing and selling the goods or services. It includes the cost of raw materials, labor, and manufacturing overhead.

	\item Selling, General and Administrative expenses (SGA). It is a category of expenses on an income statement that includes all the costs associated with running a business that are not directly related to the production of goods or services. These expenses are also known as overhead expenses.
	\begin{itemize}
		\item Selling expenses are the costs incurred to promote and sell a company's products or services. Examples of selling expenses include advertising, sales commissions, and marketing expenses.
		\item General expenses are the costs incurred to run the overall operations of a business. Examples of general expenses include rent, office supplies, and insurance.
		\item Administrative expenses are the costs incurred to manage and administer a business. Examples of administrative expenses include salaries, legal and accounting fees, and office equipment.
	\end{itemize}

    \item Gross profit: This is calculated by subtracting the COGS from the revenues. It represents the profit a company makes before accounting for other expenses.

    \item Operating expenses: This section includes all the expenses that a company incurs while running its business, such as rent, utilities, salaries, and marketing costs.

    \item Operating income: This is calculated by subtracting the operating expenses from the gross profit. It represents the profit a company makes from its main operations.

    \item Other income and expenses: This section includes any other income or expenses that do not fall into the categories above. It includes things such as interest income or loss from investments.

    \item Net income: This is calculated by subtracting total expenses from total revenues. It represents the profit or loss of a company over a specific period of time.

\end{itemize}

\textcolor{red}{An income statement does not typically include taxes as an explicit line item.}

The income statement is an important financial statement for evaluating a company's financial performance over a period of time and can help to identify trends in revenue, expenses, and net income. It provides information that can be used to make predictions about the company's future performance and make important business decisions.

\section{Conclusion}

Your conclusion text goes here.

\end{document}
