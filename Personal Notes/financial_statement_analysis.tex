\documentclass{article}
\usepackage{amsmath} % for equations
\usepackage{xcolor}
\usepackage[top=0.75in, bottom=0.75in, left=0.75in, right=0.75in]{geometry}

\title{Introduction to Fundamental Analysis}
\author{John Forrester Davidson}
\date{} % leave date blank
\begin{document}
\maketitle
\tableofcontents

\section{Introduction}

These notes aim to cover the fundamental aspects of financial analysis. The focus will be on the balance sheet, income statement, and cash flow statements, which are the main financial statements used to evaluate a company's financial performance and position. Additionally, there will be sections on stocks and bonds, which are the main types of securities that companies issue to raise capital. These sections will cover what stocks and bonds represent, and how to value these securities. The objective of these notes is to provide a comprehensive understanding of the key concepts and tools used in financial analysis, which will enable the reader to effectively analyze a company's financial statements and make informed investment decisions.

\section{Balance Sheets}

We begin by looking at the balance sheet, which is a snapshot of a company's financial position at a single point in time. It reports the company's assets, liabilities, and equity as of a specific date. It is one of the most important financial statements for understanding a company's overall financial health. It helps to understand the company's liquidity, its ability to pay off its debt, and the amount of assets it owns compared to how much it owes. The balance sheet provides a detailed picture of what a company owns (assets) and what it owes (liabilities) at a specific point in time. By comparing the balance sheet of a company from one period to another, you can understand how the company's financial position has changed over time. The basic accounting equation to consider is the following: 

\begin{equation}
    Assets = Liabilities + Owners Equity
\end{equation}

The intuition behind this equation is as follows. A company's assets are financed either by borrowing money (liabilities) or by using the funds of its owners (equity). In other words, the assets of a company are the result of the funding provided by its creditors and its owners. Assets represent the resources the business has available to generate income and grow the business. Liabilities, on the other hand, represent the obligations or debts that the company has to its creditors. Owners' equity represents the residual interest in the assets of the company after liabilities are deducted, it represents the portion of the assets that the company's owners can claim as their own. It helps to understand that the assets of the company are the means to generate revenue, the liabilities are the obligations to pay back to others and the owners equity are the funds that the owners have invested in the company and what they are left with.

\subsection{Assets}

Long-term assets and current assets are two types of assets that are typically found on a company's balance sheet.\\

Current assets are assets that are expected to be converted into cash or used up within one year of the balance sheet date. Examples of current assets include cash, accounts receivable, and inventory. These assets are considered to be liquid, as they can be easily converted into cash to pay off short-term debts or meet other short-term obligations.\\

Long-term assets, on the other hand, are assets that are not expected to be converted into cash or used up within one year of the balance sheet date. Examples of long-term assets include property, plant, and equipment (PPE), investments in other companies, and intangible assets such as patents and trademarks. These assets are considered to be less liquid than current assets, as they are not as easily converted into cash. Note that inventory may under some circumstances take a while to be sold, and if liquidation occurs then often inventory is sold as a steep discount. \\

Property, Plant and Equipment (PPE) is a type of long-term asset that includes tangible assets used in the operations of a business, such as land, buildings, machinery, vehicles and office equipment. This type of assets are used in the production of goods or provision of services and are not intended for resale. PPE typically has a useful life of more than one year, and is recorded at cost and depreciated over time.

\subsection{Liabilities}

The liabilities section of a balance sheet lists a company's outstanding debts and financial obligations as of a specific date. It represents the money that the company owes to its creditors and other parties. The liabilities section is typically divided into two categories: current liabilities and long-term liabilities.\\

Current liabilities are debts and obligations that are due within one year of the balance sheet date. Examples of current liabilities include accounts payable, short-term loans, and taxes owed. These are debts that the company expects to pay off within the next 12 months using current assets such as cash, accounts receivable, and inventory.\\

Long-term liabilities are debts and obligations that are due beyond one year of the balance sheet date. Examples of long-term liabilities include long-term loans, bonds, and leases. These are debts that the company expects to pay off over a period of time greater than one year, typically with the help of long-term assets such as PPE and investments.\\

It's important to note that the liabilities section of the balance sheet not only includes debt but also any other financial obligations. For example, it may include things like rent or salaries that are not yet paid, or taxes that are owed.\\

The liabilities section of the balance sheet is important for understanding a company's financial position and its ability to pay off its debts. A company with a lot of liabilities may have a harder time paying off its debts than a company with fewer liabilities. Additionally, the liabilities section can provide insight into the company's financial strategy and risk management policies. A company with a lot of short-term liabilities may be taking on more risk than a company with mostly long-term liabilities.

\subsection{Owners Equity}
The owners' equity portion of a balance sheet represents the residual interest in the assets of the company after deducting liabilities. It represents the value of the assets that the company's shareholders would be entitled to if the company were to be liquidated and all of its assets sold and all of its liabilities paid off. It is also known as stockholders' equity, it is the value of the shareholders' interest in the company.

The owners' equity section of the balance sheet typically includes the following accounts:

\begin{itemize}
    \item Common Stock: This represents the number of shares of stock a company has issued to its shareholders and the par value of those shares.
    \item Paid-in Capital: This represents the amount of money shareholders have invested in the company in exchange for stock, in excess of the par value.
    \item Retained Earnings: This represents the portion of net income that the company has chosen to retain and not distribute as dividends.
    \item Treasury Stock: This represents the shares of stock that a company has bought back from its shareholders.
    \item Accumulated Other Comprehensive Income: This represents other income and expenses that are not included in net income, such as foreign currency translation adjustments or unrealized gains or losses on securities.
\end{itemize}

The owners' equity section of the balance sheet is important for understanding a company's financial position and its ability to raise capital. It can provide insight into how much money shareholders have invested in the company and how much the company has earned over time. Additionally, it can provide insight into a company's dividend policy and its approach to stock buybacks.

Note that owners equity is not a liability, as it is not owed to anybody except in the sense that the company is "owed to" or belongs to the shareholders. 

\subsubsection{A Note on Book Value}
Book value is a financial metric that is calculated by taking the total assets of a company and subtracting its total liabilities. It represents the value of a company's assets if they were to be liquidated at their current market value.

In some cases, the book value of a company can be equal to its owners' equity. This typically happens when a company has few liabilities and its assets are valued at their historical cost rather than their market value. However, this is not always the case. The book value can be different from the owners' equity, especially when the company's assets are valued at market value rather than historical cost, this will cause the book value to be different from the owners' equity.

It's important to note that Book value and owners' equity are not the same thing, but they are related. The book value is a measure of the company's net asset value, whereas the owner's equity is a measure of the shareholder's stake in the company.



\section{Income Statements}

An income statement, also known as a profit \& loss statement (P\&L), is a financial statement that shows a company's revenues and expenses over a specific period of time, typically a fiscal quarter or year. The income statement is used to calculate a company's net income, which is the difference between its total revenue and total expenses. It helps to understand how much a company is making or losing over a period of time. An income statement typically includes the following sections:

\begin{itemize}
    \item Revenues: This section includes all the money a company earns from selling goods or services. It includes both cash and credit sales.

    \item Cost of goods sold (COGS): This section includes all the costs directly associated with producing and selling the goods or services. It includes the cost of raw materials, labor, and manufacturing overhead.

	\item Selling, General and Administrative expenses (SGA). It is a category of expenses on an income statement that includes all the costs associated with running a business that are not directly related to the production of goods or services. These expenses are also known as overhead expenses.
	\begin{itemize}
		\item Selling expenses are the costs incurred to promote and sell a company's products or services. Examples of selling expenses include advertising, sales commissions, and marketing expenses.
		\item General expenses are the costs incurred to run the overall operations of a business. Examples of general expenses include rent, office supplies, and insurance.
		\item Administrative expenses are the costs incurred to manage and administer a business. Examples of administrative expenses include salaries, legal and accounting fees, and office equipment.
	\end{itemize}

    \item Gross profit: This is calculated by subtracting the COGS from the revenues. It represents the profit a company makes before accounting for other expenses.

    \item Operating expenses: This section includes all the expenses that a company incurs while running its business, such as rent, utilities, salaries, and marketing costs.

    \item Operating income: This is calculated by subtracting the operating expenses from the gross profit. It represents the profit a company makes from its main operations.

    \item Other income and expenses: This section includes any other income or expenses that do not fall into the categories above. It includes things such as interest income or loss from investments.

    \item Net income: This is calculated by subtracting total expenses from total revenues. It represents the profit or loss of a company over a specific period of time.

\end{itemize}

Note the following distinctions carefully:
\begin{itemize}
    \item COGS is the dollar cost of the goods that have actually been sold.
    \item The dollar cost of goods that have been manufactured, but not yet sold, is put on the balance sheet as finished goods. They are not put onto the income statement as COGS until the goods have actually been sold.
    \item Labour cost that does not directly go towards the creation of goods thus cannot be attributed to any goods. It is thus not accounted for in either Finished Goods nor COGS. Rather, it becomes SGA and put on the income statement. SGA is almost never put on the balance sheet. 
\end{itemize}

\textcolor{red}{An income statement does not typically include taxes as an explicit line item.}

The income statement is an important financial statement for evaluating a company's financial performance over a period of time and can help to identify trends in revenue, expenses, and net income. It provides information that can be used to make predictions about the company's future performance and make important business decisions.


\section{Shares}
\subsection{Authorized, Issued \& Outstanding Shares}
In the context of a company's stock:
\begin{itemize}
\item "Authorized shares" refer to the total number of shares of stock that a company is legally allowed to issue, as stated in its articles of incorporation or bylaws. This number is set by the company's board of directors and can be changed by a vote of the shareholders. Authorized shares represent the maximum number of shares that can be issued and traded in the market.
\item "Issued shares" refer to the total number of shares of stock that a company has actually issued and sold to investors. This number is typically less than the number of authorized shares, as a company may choose not to issue all of the shares that it is legally allowed to.
\item "Outstanding shares" refer to the total number of shares of stock that are currently owned by investors and are available for trading in the market. This number is typically less than the number of issued shares, as a company may have issued shares that have been repurchased or retired.
\end{itemize}

\textcolor{red}{This becomes important in financial ratio calculations, as the line item "Common Stock" on the balance sheet often represents the authorized shares and not the outstanding shares.}




\section{Financial Ratios}
Financial ratios are a powerful tool used to analyze a company's financial performance and health. They are used to compare a company's financials to industry averages, to assess the company's liquidity, profitability, and solvency, and to identify trends and patterns in a company's financial performance over time. In this section, we will be discussing some of the most commonly used financial ratios in finance and accounting, such as the earnings per share, the current ratio, the quick ratio, the gross profit margin, and the return on equity. We will also explain how these ratios are calculated and what they can tell us about a company's financial performance. By understanding these financial ratios, you will be able to gain valuable insights into a company's financial health and make more informed decisions about investing in or working with the company.
\subsection{Stock Evaluation Ratios}
\subsubsection{Earnings Per Share}
Earnings per share (EPS) is a financial ratio that measures the profitability of a company by calculating the amount of net income earned for each outstanding common share of stock. It is an indicator of the company's profitability for the shareholders as it is a measure of how much money each share can earn for the investor. The price that one would be willing to pay for a stock today is not what the share of stock is earning today, but rather \textit{what it is expected to earn over a period of many years.} Stocks with a higher EPS may be willing to payout higher dividends per share. \\

EPS is calculated by dividing a company's net income by the number of outstanding shares of common stock. The formula for EPS is:

\begin{equation}
    EPS = \frac{Net Earnings - PreferredDivdends}{Common Stock Outstanding}
\end{equation}\\

It's important to note that companies may use different methods to calculate their EPS, like the basic EPS and the diluted EPS. Basic EPS uses the number of shares outstanding at the end of the period, while diluted EPS takes into account the effect of dilutive securities such as options and convertible bonds.\\

EPS is based on common stock because it is intended to measure the profitability of a company for its common shareholders, who have the most risk in the company. Common shareholders are the owners of the company and are entitled to a share of the company's profits in the form of dividends. They also have voting rights in the company, which allows them to elect the board of directors and vote on important company matters. Preferred stock, on the other hand, is a type of stock that typically has a fixed dividend and does not have voting rights. Preferred shareholders have a higher claim on the company's assets and earnings than common shareholders, but they do not have the same level of control or risk in the company.\\

For this reason, EPS is calculated using the net income available for common shareholders, which is the net income minus any dividends paid to preferred stockholders. This provides a measure of the profitability of the company for the common shareholders, who are the ones most affected by the company's performance.\\

Additionally, EPS gives a measure of the company's earning capacity per share of stock, and this is more relevant for common stock as it can fluctuate in price and the value of an individual's investment in the company depends on the number of shares they own. Preferred stock dividends are fixed and the value of the investment is not affected by the number of shares owned, thus EPS is not useful in this context.

Always be weary when doing EPS calculations. The number of outstanding shares listed on a balance sheet
\subsubsection{Price-to-Earnings Ratio}
This is as simple as it sounds. You divide the current stock price by the EPS calculated above.
\begin{equation}
    P/E = \frac{Stock Price}{EPS}
\end{equation}\\

A high P/E ratio can indicate that a stock is overvalued, while a low P/E ratio can indicate that a stock is undervalued. However, it's important to note that a high P/E ratio can also indicate that a company has high growth prospects, while a low P/E ratio can indicate that a company has low growth prospects. It's also important to consider the industry average P/E ratio when evaluating a company's P/E ratio.\\

In general, if a company's P/E ratio is greater than the industry average, it may be overvalued, and if it is less than the industry average, it may be undervalued. However, it is also important to consider other factors such as the company's growth prospects and financial health.\\

It's important to note that P/E ratio is not a perfect measure of a company's value, as it does not take into account the company's debt or cash on hand. It's also important to consider the historical trend of the P/E ratio, because a high P/E ratio today does not necessarily mean a stock is overvalued if the ratio has been trending upward over time.\\

\subsubsection{Book Value per Common Share}
Book value per common share is simply the book value, as define in section 2 above, divided by the number of outstanding shares. 
\begin{equation}
    Book\: Value\: per\: Share = \frac{Total Assets - Total Liabilities}{Outstanding Shares}
\end{equation}\\

In some cases the book value is equal to the company's shareholder equity, which is the net assets available to shareholders. Book value per share is often used as a measure of a stock's intrinsic value, as it gives investors an idea of how much a company would be worth if it were to be liquidated and its assets sold off. If a stock is trading at a price below its book value, it may be considered undervalued.

\subsubsection{Dividend Payout Ratio}
The dividend payout ratio measures the proportion of a company's earnings that are paid out to shareholders as dividends. It is calculated by dividing the total dividends paid to shareholders by the company's net income. The ratio is often expressed as a percentage.
\begin{equation}
    Dividend\: Payout\: Ratio\ = \frac{Total\: Dividends\: Paid}{Total\: Outstanding\: Shares} = \frac{Dividend\: per\: Share}{EPS}
\end{equation}\\


A high dividend payout ratio can indicate that a company is distributing a significant portion of its profits to shareholders, which can be seen as a positive sign for investors who are looking for income from their investments. A low payout ratio, on the other hand, may indicate that a company is retaining more of its earnings to reinvest in growth opportunities or to pay down debt.\\

A high payout ratio can also be a negative sign if the company is not generating enough profits to sustain its dividend payments. In this case, the company may have to cut its dividend or borrow money to pay dividends, which can be seen as a red flag for investors. Also note that companies in different industries tend to have different payout ratios; for example, utilities tend to have higher payout ratios than technology companies.

\subsubsection{Dividend Yield}
Dividend yield measures the amount of cash dividends paid out to shareholders relative to the current market price of the stock. It is typically expressed as a percentage and is calculated by dividing the annual dividend per share by the current market price per share. Most companies pay out dividends quarterly, so the annual dividend can be approximated by multiplying the most recent quarterly dividend by 4. 

\begin{equation}
    Dividend\: Yield = \frac{Dividend}{Stock\: Price}
\end{equation}\\

When speaking about the dividend yield without otherwise specifying, it generally means the expected dividend over the next year divided by the stock price today. A high dividend yield can indicate that a stock is paying out a large portion of its earnings to shareholders, which can be seen as a positive sign for investors who are looking for income from their investments. A low yield, on the other hand, may indicate that a stock is retaining more of its earnings to reinvest in growth opportunities or to pay down debt.
\subsection{Profitability Ratios}
\subsubsection{Gross Margin}
The gross profit margin (a.k.a Gross Margin) measures a company's profitability by comparing the revenue earned to the cost of goods sold. It is calculated by dividing the gross profit (revenue minus cost of goods sold) by the revenue. The result is expressed as a percentage.
\begin{equation}
    Gross\: Margin = \frac{Gross\: Profit}{Sales} = \frac{Sales - COGS}{Sales}
\end{equation}\\

A higher gross profit margin indicates that a company is more efficient at generating revenue from the sale of its goods or services, and that it has better control over its cost of goods sold. A lower gross profit margin indicates that a company is less efficient at generating revenue, and has less control over its cost of goods sold. Note that gross margins tens to vary greatly in different industries

\subsubsection{Operating Profit Margin}
Operating profit, also known as operating income, measures a company's profitability from its core operations and can be used as a measure of management's effectiveness in controlling expenses associated with normal operations. It is a measure of the efficiency of a company. It is calculated by subtracting operating expenses from revenue. Operating expenses include things like cost of goods sold, selling, general and administrative expenses, and research and development expenses. Operating income excludes non-operating items such as interest and taxes, as well as one-time items such as gains or losses from the sale of assets.
\begin{equation}
    Operating\: Profit = Sales - COGS - SG\&A - R\&D
\end{equation}\\

The operating profit margin would thus be:
\begin{equation}
    Operating\:Profit\:Margin = \frac{Operating\:Profit}{Sales} == \frac{EBIT}{Sales}
\end{equation}\\

Operating profit is sometimes referred to as EBIT, which stands for Earnings Before Interest \& Taxes. Interest and taxes are not considered operating expenses. Operating margin improvement, known as margin expansion, is generally a favourable thing for a company. Margin expansion can imply:
\begin{itemize}
\item The company has increased sales with a smaller percentage increase in costs.
\item Management was able to raise prices without losing business.
\item Management has found a way to reduce costs.
\end{itemize}

Regarding the first of the above items, most businesses have some relatively fixed costs, so margin expansion due to increasing sales while some costs remain fixed is referred to as \textit{fixed cost leverage} or \textit{manufacturing leverage}. To the extent that the fixed costs were a part of SG\&A, we would say the margin expansion is due to \textit{SG\&A leverage}

\subsubsection{Pretax Profit Margin}
The pretax profit margin is also called the pretax return on sales. It is the profit before taxes divided by the sales for the same period. 
\begin{equation}
    Pretax\: Profit\: Margin = \frac{Pretax\: Profit}{Sales}
\end{equation}\\

While operating profit is sales minus all operating expenses (COGS, SGA, R\&D), pretax profit includes what are called \textit{non-operating expenses}. Pretax profit thus includes things like interest expenses (or interest income), gains or losses from investments, foreign exchange gains or losses, and other one-time items. 

Like the operating margin, it is also a measure of the efficiency of a company. Two companies of the same size and same industry can be compared in efficiency via their pretax profit margins. This may not be true if the companies are of different sizes. A larger company can spread fixed overhead costs over more units sold. The lower unit cost would likely create larger profit margins. Often the company with the highest pretax profit margin in an indiutry is the safest of all companies in that industry. 

\subsubsection{Net Profit Margin}
The net profit margin is calculated by deducting taxes from the pretax profit and then dividing by sales. It means essentially the same thing as the pretax profit margin. 

\subsubsection{Return on Invested Capital (ROIC)}
Return on Invested Capital is a measure of how well a company is investing its capital. \textcolor{red}{For now we define invested capital (total capitalization) as the amount of long term debt plus equity carried on the balance sheet.} Recall that capitalization may be thought of as the sources of money that bought the capital assets. Therefore, the return on capital is a measure of how the company is able to use its assets to generate profit. 

 \begin{equation}
    ROIC = \frac{Net\: Profit\: after\: Taxes}{Total\: Capitalization}
\end{equation}\\

Note that the capital earned by the end of a year is really earned due to the total capitalization of the company at the beginning of the year. As such, the ROIC for a company in a given year is best calculated using the total capitalization at the end of the previous year. Some people however believe that the average total capitalization throughout the year should be used, i.e. dividing the mean of the current and previous years capitalization. This is because in reality the total capitalization of a company is continually changing due to profits earned, losses incurred and money raised, debt repayments, dividend payments, stock buybacks, etc. 

ROIC, or simply return on capital, is a measure of the operating profit generated using the capital that was provided by both the debt and equity holders. A ratio of 0.05 indicates that \$0.05 is generated \textit{as after-tax operating income} for every \$1.00 in capital. 

This ratio can be used to compare companies in the same industry. Regardless of company size, a company with a larger ROIC is the fastest growing in proportion to its own equity base. ROIC tends to be lower in competitive industries such as utilities. 

\subsubsection{Return on Equity (ROE)}
eturn on Equity looks at the profitability of a company from the perspective of the shareholders alone. To calculate the ROE, the net income is divided by the \textit{book value} of the shareholders equity.
 \begin{equation}
    ROE = \frac{Net\: Income}{B.V\: of\: Shareholder's\: Equity}
\end{equation}\\
This is a closely watched metric as it is a measure of the return on equity provided specifically by shareholders. A ratio of 0.05 indicates that \$0.05 is generated \textit{as after-tax operating income} for every \$1.00 in equity. 

\subsubsection{Return on Assets (ROA)}
A related measure to ROE is the Return on Assets. This measures how management is using a companies \textit{assets} to generate income. Is is calculated similarly to the above two metrics, as follows :
 \begin{equation}
    ROA = \frac{Net\: Profit}{Total\: Assets}
\end{equation}\\

A ratio of 0.05 indicates that \$0.05 is generated \textit{as after-tax operating income} for every \$1.00 in assets. This is a good metric to use to compare companies in the same industry or to watch one company over time. It is seldom useful when comparing companies in different industries, as different industries generally require different levels of assets.


\subsection{Debt and Interest Ratios}
Leverage can be used to mean debt. A company with high debt relative to equity on the balance sheets is said to be highly leveraged. High leverage implies a large interest expense on the outstanding debt, and can therefore mean that a company is more risky is there is any doubt in the company's ability to make the interest payments. \\

The first two metrics to follow deal directly with a company's ability to raise more money if needed. 

\subsubsection{Interest Coverage Ratio}
The interest coverage ratio (ICR) is sometimes called the times-interest-earned or the earnings coverage ratio. It measures a company's ability to pay the interest on its outstanding debt. It is calculated by dividing a company's earnings before interest and taxes (EBIT) (\textit{operating profit, the money earned that is available to pay interest charges}) by its total interest expense for a specific period of time, typically a year. \\

 \begin{equation}
    ICR = \frac{EBIT}{Total\: Interest\: Expense}
\end{equation}\\

An ICR of 6x indicates that a company earns 6x the amount required in order to pay back its interest expenses. Larger values are indicate that the company would more easily be able to pay back its debt and would have an easier time borrowing additional capital. Note that taxes do not enter the equation, as interest charges are deducted before pretax profit and taxes are calculated. 

The ICR answers two questions:
\begin{itemize}
\item How much money is the company earning that is available to pay interest?
\item How many times larger is the available amount of earnings than it needs to be?
\end{itemize}

Since interest is paid with cash, we could simply look at the cash on the balance sheet to determine if the company has sufficient funds. But this is generally not a good indicator as it only represents the available cash at a single point in time. We are more interested in whether the company has sufficient cash flows over time to pay the required interest charges. We thus generally look at the income statement


\subsubsection{Fixed Charge Coverage Ratio}
The fixed charge coverage ratio (FCCR) measures a company's ability to meet its fixed financial obligations, such as interest payments and lease payments. This is calculated much like the interest coverage ratio, but includes other fixed charges such as fixed lease payments. The ratio is calculated by dividing a company's earnings before interest, taxes, depreciation and amortization (EBITDA) by its fixed charges, which include interest expenses and lease payments.

 \begin{equation}
    FCCR = \frac{EBITDA}{Total\: Fixed\: Charges}
\end{equation}\\

It is important to note that the EBITDA is used as the numerator in the calculation, because it excludes the non-cash expense of depreciation and amortization, which allows for a better comparison of a company's operating performance over time. FCCR is similar to the interest coverage ratio (ICR) but it gives a more comprehensive picture of a company's ability to meet its fixed financial obligations. While ICR only considers the interest expenses, FCCR takes into account both interest expenses and lease payments.


\subsubsection{Debt to Total Capitalization (D/TC)}
Debt to total capitalization, sometimes referred to as \textit{debt to total capital} or \textit{debt to cap} usually refers to long-term debt divided by total capitalization. Recall that total capitalization is the sum of a company's debt and equity, which represents the total value of the company's capital structure.

 \begin{equation}
    D/TC = \frac{Long\: Term\: Debt}{Total\: Capitalization}
\end{equation}\\

The less debt already in total nationalization, the easier (i.e. at lower interest) the company can borrow more money. Typical ratio values vary widely across industries, as some industries are more capital-intensive and may have a higher D/TC ratio than others. Therefore, it's important to compare the ratio with other companies in the same industry to determine if a company's leverage is relatively high or low.

\subsubsection{Total Debt Ratio}
The total debt ratio indicates how much debt a company has relative to its assets. The ratio is calculated by dividing a company's total debt by its total assets. 

 \begin{equation}
    Total\: Debt\: Ratio = \frac{Total\: Assets\: - Total\: Equity}{Total\: Assets}
\end{equation}\\

Unlike the debt-to-cap ratio above, the total debt ratio includes current liabilities such as Accounts Payable. The total debt ratio may thus flag issues that the debt-to-cap ratio might miss, e.g. if current liabilities were abnormally high. Generally speaking, companies with total debt ratios larger than 0.5 are considered to be highly leveraged, but typical ratio values are highly industry dependent. 

\subsection{Liquidity \& Financial Condition Ratios}
The significance of the next three ratios lies not in the values of the actual ratios, as these vary widely between industries and even between companies in the same industry. The focus is more on the changes that occur over a period of time within the same company. The changes in the ratios over time give an indication of the trajectory of a company's financial health. 

\subsubsection{Current Ratio}
The current ratio is defined as current assets divided by total liabilities. It measures a company's ability to pay off its short-term liabilities. Current assets are assets that can be easily converted into cash within a year, such as cash, marketable securities, accounts receivable and inventory. Current liabilities are obligations that are due within a year, such as accounts payable, short-term debt and taxes.

 \begin{equation}
    Current\: Ratio = \frac{Current\: Assets}{Current\: Liabilities}
\end{equation}\\

A higher ratio indicates that a company has a higher level of liquidity and is more able to meet its short-term obligations. A lower ratio, on the other hand, indicates that a company has a lower level of liquidity and may have difficulty meeting its short-term obligations. Creditors generally prefer ratios that refer high liquidity. However, a high current ratio may indicate poor use of cash or poor inventory control procedures. 

\subsubsection{Quick Ratio}
The quick ratio is another measure of a company's ability to pay off debt in the short-term. It is sometimes referred to as the acid test ratio. Quick assets are regarded as those current assets that can be converted into cash quickly, such as cash, marketable securities and accounts receivable, and is almost always equal to total current assets minus inventory. The metric is calculated as follows:

 \begin{equation}
    Quick\: Ratio = \frac{Current\: Assets - Inventory}{Current\: Liabilities} = \frac{Quick\: Assets}{Current\: Liabilities}
\end{equation}\\

The quick ratio is considered to be a more stringent measure of liquidity than the current ratio as it excludes inventory, which may not be as easily converted into cash as other current assets. A large inventory may indicate that a company has overbought or that it is struggling to make sales. 

\subsubsection{Cash Ratio}
This is simply a more stringent version of the quick ratio (acid test)

\begin{equation}
    Cash\: Ratio = \frac{Cash + Marketable Securities}{Current\: Liabilities} 
\end{equation}\\

It is considered to be the most conservative measure of liquidity because it only takes into account a company's most liquid assets (i.e., cash and cash equivalents) in relation to its current liabilities.

\subsection{Efficiency Ratios}
Efficiency ratios, also known as activity ratios, are financial metrics that measure a company's ability to manage its resources and operations effectively. These ratios provide insight into how efficiently a company is using its assets, such as inventory and accounts receivable, to generate revenue. They also help to assess the company's overall financial health and performance. These ratios can sometimes give early indications that a company has a problem.

\subsubsection{Total Asset Turnover}
Total asset turnover measures a company's ability to generate revenue from its total assets. It is calculated by dividing the company's net sales by its average total assets. Note that average assets is used due to the fact that a company's total assets is generally changing across the year. 

 \begin{equation}
    Total\: Asset\:Turnover = \frac{Sales}{Total\: Assets} 
\end{equation}\\

An asset turnover of 0.7 indicates that for every dollar of assets, 0.7 dollars of sales is generated. The higher the the ratio, the more efficient the company is at using its assets to generate sales. 

\subsubsection{Inventory-to-Sales Ratio}
A company's inventory is constantly turning over. A company generally wants enough inventory at hand to cater to customers orders, but not too much as it cost money to carry a larger-than-required inventory. The inventory-to-sales ratio measures a company's efficiency in managing its inventory. The best way to watch the inventory level is to look at it in elation to the company's sales. 

\begin{equation}
    Inventory\:to\:Sales\: Ratio = \frac{Inventory}{Sales} 
\end{equation}\\

Note that, since inventory is always changing, it may be more accurate to calculate the above ratio by using average levels of company inventory. \\

Watching the ratio change over time may lead to useful insights regarding the company and its operations. Unexpected inventory build-ups may indicate declining sales, or large cancellations of orders. The reasons for such changes should always be investigated, as some may be justified while others may indicate problems. 

\subsubsection{Inventory Turnover Ratio}
You may prefer to invert the inventory-t-sales ratio and instead calculate the inventory turnover ratio as follows:

\begin{equation}
    Inventory\: Turnover Ratio = \frac{Sales}{Inventory} 
\end{equation}\\

If a company's inventory turnover ratio dropped from 3 to 2, we would say that the \textit{inventory turns} slowed sharply.\\

\textcolor{red}{When calculating either inventory-to-sales or the inventory turnover ratio, some prefer to use COGS as opposed to sales. This is because sales may have changed just because the company's selling price for the inventory has changed, which may distort the ratio and seem to indicate that a disproportionately large amount of inventory has been sold. Using COGS will prevent this distortion. }

\subsubsection{Inventory Turnover in Days}
Another way to analyse inventory is to determine the number of days' sales currently held in inventory. This is calculated by dividing 365 days by the inventory turnover ratio and tells you the average number of days that an item sits on the shelf before being sold.

\begin{equation}
    Inventory\: Turnover\: in\: Days = \frac{365\: days}{Inventory\: Turnover} = Days
\end{equation}\\

Note that if quarterly figures are being calculated, then the number of days in a quarter must be used. 

\subsubsection{Accounts Receivable to Sales Ratio}
This ratio can help give an indication that something is wrong at a company. It is calculated as follows:

\begin{equation}
    Accounts\: Receivable\: to\: Sales = \frac{Accounts\: Receivable}{Sales} 
\end{equation}\\

If the ratio were to quickly increase then this would indicate that customers are not paying their bills, which could lead to issues in payment of the company's liabilities, such as interest payments. If this is not under good control then the company could be headed to bankruptcy. 

\subsubsection{Accounts Receivable Turnover and Days Sales in Receivables}
Receivables turnover is the inverse of the accounts receivable to sales ratio, describes how quickly the company collects on those sales. 

\begin{equation}
    Receivables\: Turnover= \frac{Sales}{Receivables} 
\end{equation}\\

Using the number of days worth of sales in the above receivables ratio, we can calculate the umber of days it takes the company to collect the credit on sales. This is sometimes referred to as the \textit{average receivables collection period}, or the \textit{day sales outstanding}

\begin{equation}
    Receivables\: Turnover\: in\: Days = \frac{365\: days}{Receivables\: Turnover} = Days
\end{equation}\\


If a large increase in Days Sales Outstanding is observed, then it may be a warning sign that the company's customers are delaying payments or in some cases stopping payments. This may indicate that the customers are having financial trouble and may go out of business, cutting off a stream of revenue to the company being analysed. It may also be possible that sales for the analysed company were weak, and that they attempted to gain sales by doing business with a riskier client. 

















\section{Conclusion}

Your conclusion text goes here.


\begin{equation}
    F = ma = m\frac{d^2x}{dt^2} = m\left(\frac{\partial^2x}{\partial t^2} + \frac{\partial x}{\partial t}\frac{\partial v}{\partial x} + v\frac{\partial v}{\partial x}\right)
\end{equation}

\textcolor{red}{This becomes important in financial ratio calculations, as the line item "Common Stock" on the balance sheet often represents the authorized shares and not the outstanding shares.}

\end{document}
